%  template.tex for Biometrics papers
%
%  This file provides a template for Biometrics authors.  Use this
%  template as the starting point for creating your manuscript document.
%  See the file biomsample.tex for an example of a full-blown manuscript.

%  ALWAYS USE THE referee OPTION WITH PAPERS SUBMITTED TO BIOMETRICS!!!
%  You can see what your paper would look like typeset by removing
%  the referee option.  Because the typeset version will be in two
%  columns, however, some of your equations may be too long. DO NOT
%  use the \longequation option discussed in the user guide!!!  This option
%  is reserved ONLY for equations that are impossible to split across 
%  multiple lines; e.g., a very wide matrix.  Instead, type your equations 
%  so that they stay in one column and are split across several lines, 
%  as are almost all equations in the journal.  Use a recent version of the
%  journal as a guide. 
%  
\documentclass[useAMS,referee]{biom}
%documentclass[useAMS]{biom}
%
%  If your system does not have the AMS fonts version 2.0 installed, then
%  remove the useAMS option.
%
%  useAMS allows you to obtain upright Greek characters.
%  e.g. \umu, \upi etc.  See the section on "Upright Greek characters" in
%  this guide for further information.
%
%  If you are using AMS 2.0 fonts, bold math letters/symbols are available
%  at a larger range of sizes for NFSS release 1 and 2 (using \boldmath or
%  preferably \bmath).
% 
%  Other options are described in the user guide. Here are a few:
% 
%  -  If you use Patrick Daly's natbib  to cross-reference your 
%     bibliography entries, use the usenatbib option
%
%  -  If you use \includegraphics (graphicx package) for importing graphics
%     into your figures, use the usegraphicx option
% 
%  If you wish to typeset the paper in Times font (if you do not have the
%  PostScript Type 1 Computer Modern fonts you will need to do this to get
%  smoother fonts in a PDF file) then uncomment the next line
%  \usepackage{Times}

%%%%% PLACE YOUR OWN MACROS HERE %%%%%

\def\bSig\mathbf{\Sigma}
\newcommand{\VS}{V\&S}
\newcommand{\tr}{\mbox{tr}}

\title[Empirical Bayes analysis for detection of gene heterosis in RNAseq data]{Empirical Bayes analysis for detection of gene heterosis in RNAseq data}

\author{Jarad Niemi $^*$\email{niemi@iastate.edu}, 
Eric Mittman, 
Will Landau, and 
Dan Nettleton \\
Department of Statistics, Iowa State University, Ames, Iowa, U.S.A.}

\begin{document}

\date{{\it Received December} 2014} 

\pagerange{\pageref{firstpage}--\pageref{lastpage}} 
\volume{VV}
\pubyear{YYYY}
\artmonth{Month}
\doi{}

\label{firstpage}

%  put the summary for your paper here

\begin{abstract}
This is the summary for this paper.
\end{abstract}

\begin{keywords}
Empirical Bayes; Heterosis; Hierarchical Model; Negative binomial; RNAseq.
\end{keywords}

\maketitle

%  A maximum of six (6) tables or figures combined is often required.''

\section{Introduction}
\label{s:intro}

\section{Heterosis}
\label{s:heterosis}

\section{Hierarchical model}
\label{s:model}

\section{Empirical Bayes}
\label{s:inf}

\section{Simulation study}
\label{s:simulation}

\section{Heterosis in corn ??}
\label{s:corn}

\section{Discussion}
\label{s:discuss}

Put your final comments here. 



\backmatter %  Please keep this command in your document in this position 



\section*{Acknowledgements}

The authors thank Andrew Lithio for help implementing our model in {\tt ShrinkBayes}.

%  If your paper refers to supplementary web material, then you MUST
%  include this section!!  See Instructions for Authors at the journal
%  website http://www.biometrics.tibs.org

\section*{Supplementary Materials}

Web Appendix A, referenced in Section~\ref{s:model}, is available with
this paper at the Biometrics website on Wiley Online
Library.\vspace*{-8pt}

\bibliography{jarad}
\bibliographystyle{biom}

\appendix

%  To get the journal style of heading for an appendix, mimic the following.

\section{}
\subsection{Stan model for heterosis}



\label{lastpage}

\end{document}
