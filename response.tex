\documentclass{article}

\newcommand{\comment}[1]{\textit{#1}}
\newcommand{\response}[1]{#1}

\begin{document}


Editor comments:

\comment{The proposed model appears to be an extension of Ji et al. (2014) albeit with a differently specified likelihood function (negative binomial or Poisson-gamma) as perceived to be more appropriate for count data relative to the Gaussian likelihood function pursued in Ji et al. (2014). Both reviewers indicate that they are little underwhelmed by the performance of your proposed procedure relative to some of the existing methods that you compare against (i.e., ShrinkBayes, edgeR and baySeq).  Furthermore, they seem somewhat concerned about an ‘apples vs. oranges’ types of comparisons since the proposed hypothesis testing strategy in the proposed method is really not an option in the other methods.
Perhaps a reference to the promise of allele specific expression analyses (Discussion section) for inferring upon gene-expression heterosis might be useful to point readers into other directions with this work (Bell et al., 2013; Wei and Wang, 2013).}

\comment{I realize that Ji et al.’s (2014) work was intended for microarray data, or more specifically Gaussian likelihoods, but there has been some work that suggests that Gaussian likelihood specifications in MAANOVA may, oddly enough, have better properties for the analysis of RNA-seq data than more popular methods specifically intended for count data (edgeR and DESeq).  See Reeb and Steibel (2013) and also comments from one reviewer along these lines.  Since Ji et al.’s work also derives from ISU, have you assessed its performance relative to your proposed method on RNA-seq data?}

\comment{One seemingly glaring omission from this work is any comparative assessment of the proposed versus other methods (e.g. ShrinkBayes) on the actual maize dataset in Section 4.  Why is that missing?
The comments made by one reviewer regarding a relative assessment of computing time relative to the other methods deserves attention.}

\comment{Page 1.  I was instructed that heterosis is merely characterized by the hybrid performance not being equal to the average of the two parental lines.  So in your very first sentence, I might write “Heterosis, or hybrid vigor, occurs when hybrid progeny display phenotypes that are superior to the average phenotypes of their parents.” i.e., I might be only interested in the $\delta_g$’s. However, your later characterization of heterosis is really something called overdominance to me…where the hybrid performance exceeds the performance of either parent.  I wonder if you could then be more specific in your definitions in your first sentence…and also in your third sentence in this particular paragraph.}

\comment{Page 1 lines 48-59.  Perhaps this was already captured well in the Ji et al. (2014) paper but why not simply test heterosis as the difference between the hybrid and the average parental performance (i.e. $\mu_3 – 0.5 \mu_1 -0.5 \mu_2$).  Again, I think your issue pertains really to an overdominance test of some sort.}

\comment{Page 2 line 35.  “We CONSIDER an…”}

\response{Fixed.}

\comment{Page 3 Line 7.  I don’t it is sufficient to mention the statistical software “Stan”.  Please provide a suitable description and citation.}

\comment{Page 3 Line 26.  There are several ways to construct a Poisson-gamma distribution, all leading to different specifications of the relationship between the mean and variance of a negative binomial.  I apologize for the admittedly vain self-citation but there is one specification, for example, that specifies the variance to be a linear function of the mean of a negative binomial…see Equation 2.12b in  Tempelman and Gianola (1996).  I guess I’m not sure why most RNA-seq data analysts gravitate to a specification where the variance is specified to be linear and quadratic function of the mean as you do here.}

\comment{Page 3. Bottom.  Shouldn’t you at least provide a justification when one would use a normal prior and when one would or should use a Laplace prior?}

\comment{Page 4, bottom half.  The strategy described here, also somewhat attested to by a reviewer, is a little disconcerting with respect to the “double estimation” of gene-specific parameters…first to estimate they hyperparameters and then again conditioned on the estimates of these hyperparameters.  Perhaps you should characterize that these second-stage gene-specific estimates are shrunk relative to those estimates used to estimate the hyperparameters and that they then have lead to predictions with better properties?  Ideally, I would have liked to see these hyperparameters be also formally estimated together with the gene specific parameters within one MCMC analyses…authors should address why they didn’t do this.}



\comment{Page 5, lines 37-47.  Since you’re doing MCMC, why not simply tabulate the number of times samples of $\mu_3$ exceeds either $\mu_1$ or $\mu_2$….and/or the number of times that $\mu_3$ is less than $\mu_1$ or $\mu_2$.  I suspect that what your test is really doing….in which case it might be useful to explicitly state this equivalence as such.}

\comment{Page 6, top.  Is it possible then to assess your proposed method using Laplace versus normal priors to assess if that seems to be contributing to any differences between ShrinkBayes and your proposed method?  Maybe that is the real reason for the small improvements?...., and not the differences in sophistication of the constructed statistical tests?}

\comment{Page 7, line 15.  Why delete the low counts?  Data quality issues?}

\comment{Page 7, lines 17-22.  Can you provide us any information on how well the hyperparameters were estimated based on your empirical Bayes strategy?}

\comment{Page 8, line 37.  It is not obvious to me why $H_g2*$ (are the two parents the same?) is a necessary component of a heterosis (overdominance) test!?}

\comment{Page 9, lines 23-25.  In tandem with another reviewer’s comments, the real question is why?  Is it because of the suboptimality of the test that you contrived for ShrinkBayesA?}

\comment{Page 10, Line 29.  Don’t you mean “nadir” rather than peak?}

\comment{Page 11 line 39.  Is it really “independence” or nearly non-identifiability that is the issue?}

\comment{Page 11 last paragraph.  So this then begs an assessment of how good/poor these hyperparameter estimates were.}

\comment{Page 11 last line “ASYMPTOTIC”}}

\response{Fixed.}

Reviewer \#1

\comment{The authors develop a hierarchical negative binomial model for drawing inferences from a composite null hypothesis, that tests gene expression heterosis. Simulations are performed to describe the performance of this new method along with a data analysis demonstrating an application to real data. This manuscript represents a well-conceived and executed methodological development in the statistical analysis of RNA-Seq gene expression data. However, the performance of the new method is underwhelming compared to existing methods, tempering my enthusiasm for the likely impact of this paper.}

\comment{(1) In the simulations, the authors should also compare the eBayes method to the method of Ji et al. (2014). Although developed for continuous microarray data, the authors could apply a simple transformation to the count data in order to make them continuous. I would be curious to know whether the eBayes method out-performs the method of Ji et al. (2014) on the transformed data.}

\comment{(2) The authors should provide more details on the computational price of the eBayes method? How long (in comparison to edgeR, baySeq, ShrinkBayes) did the eBayes method take to run? Are these methods comparable in terms of computational cost?}

\comment{(3) In the description of the simulation results, the eBayes and ShrinkBayes methods display very similar performance. The distinction between the results on the Figures is quite minimal. The authors should attempt to quantify the difference in performance in a single metric. For instance, in Figure 1, take the average percent difference in TPR across the FPRs that were used to generate the plot.}

Reviewer \#4

\comment{1. The authors propose a method to assess "gene expression heterosis."   While it is mentioned that this type of heterosis is a possible explanation for phenotypic heterosis, I'd like more discussion regarding this point.  The two references seem a bit outdated (for bioinformatics) so I'm wondering if this explanation is still being considered.  If it is, than I'd like to see more discussion about how the analytic results would be used to investigate this explanation.   Wouldn't scientists have specific phenotypic traits in mind and know of genes associated with these traits?  I'm a little confused why the search through the whole genome for this type of heterosis if they're looking to draw inference on the association.}


\comment{2. The model appears reasonable and is adapted from on used with microarrays.  To support their approach, they simulate data using their model (ideal case) and find that their approach does slightly better than ShrinkBayes.}

\comment{a. How robust do you feel your method is?  There is no discussion along these lines.}

\comment{b.      Why is there more of an improvement over ShrinkBayes with more replicates?  Is it because of the method used to calculate the posterior probabilities?  I'd like to see some discussion on this.}

\comment{c.      The ROC curves are not particularly impressive.  Even with 16 reps, the true positive rate is below 50\% (controlling for the false positive rate).  The real data example included only 4 reps, which is far more likely a scenario.  Very little discussion is devoted to this.}

\comment{3.      Minor edits or comments}

\comment{a.      Start of Section 2 on page 2  :  Replace "We considering" with just "Consider"}

\response{Fixed.}

\comment{b.      Page 3: "The parental averages and overdispersion parameters are assumed to follow normal distributions, i.e.,"}

\response{Fixed.}

\comment{c.      Right before Section 2.2…it is assumed apriori independence, correct?}

\response{Yes, added \emph{a priori}.}

\comment{d.      Page 4: You have a two-step approach where you estimate the parameters of interest to get estimates of the hyperparameters and then restimate.  How much change is there is these estimates?  Why aren't the methods of McCarthy used as a competitor?}

\comment{e.      Page 11 "These figures show marked departures from ……"}

\response{Fixed.}

\comment{f.      Page 11 : Does it really matter that the posterior is different from the prior?  One often assume aprior independence but expects there to be posterior correlation.  Perhaps I'm just missing something in this discussion.}


\end{document}
